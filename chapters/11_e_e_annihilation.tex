\chapter{\Pelectron \Ppositron annihilation}
The annihilation process \HepProcess{\Pelectron \Ppositron \to \Pmuon \APmuon} is related to the scattering process \HepProcess{\Pelectron \Pmuon \to \Pelectron \Pmuon} as can be seen by comparing their diagrams. The diagram for the annihilation process is
\begin{figure}[th]
\centering
\unitlength = 1mm
\begin{fmffile}{annihilate}
    \begin{fmfgraph*}(65,20)
        
        \fmfleftn{i}{2}
        \fmfrightn{o}{2}
        
        \fmf{fermion,label=\Pmu}{o1,v1,o2}
        \fmf{fermion,label=\Pe}{i2,v2,i1}
        \fmf{photon,label=\Pphoton}{v1,v2}
        
        \fmfdot{v1,v2}
        
        \fmflabel{$k$}{i2}
        \fmflabel{$p$}{i1}
        \fmflabel{$k^\prime$}{o2}
        \fmflabel{$p^\prime$}{o1}
        
    \end{fmfgraph*}
\end{fmffile}
\caption{Diagram for the s-channel electron annihilation process to positrons.}
\end{figure}
which is the same as for t-channel elastic scattering, only rotated by $90^\circ$. The rotation is equivalent to exchanging $k^\prime$ with $-p$. The consequence of this transformation is simply that the roles of $s$ and $t$ are swapped in the probability $\abs{T_{fi}}^2$, so the differential cross section for the annihilation process is
\begin{equation}\boxed{
\dv{\sigma}{\Omega} = \frac{e^4}{32\pi^2s} \frac{t^2+u^2}{s^2}
}\end{equation}

\section{Total cross section}
We wish to integrate the differential cross section in the limit where the masses of the particles involved become zero.

Write the kinematic Mandelstam variables in term of energies:
\begin{align}
s = (k + p)^2 &\approx 2k \cdot p \nonumber \\
&= 2 \mqty(E_{\Pelectron} \\ E_{\Pelectron}) \cdot \mqty(E_{\Ppositron} \\ -E_{\Ppositron}) \nonumber \\
&= 4E_{\Pelectron} E_{\Ppositron}
\end{align}
\begin{align}
t = (k - k^\prime)^2 &\approx -2k\cdot k^\prime \nonumber \\
&= -2 \mqty(E_{\Pelectron} \\ E_{\Pelectron}) \cdot \mqty(E_{\Pmuon} \\ E_{\Pmuon} \cdot \cos\theta) \nonumber \\
&= -2 E_{\Pelectron}E_{\Pmuon}(1-\cos\theta)
\end{align}
\begin{align}
u = (k-p^\prime)^2 &\approx -2k\cdot p^\prime \nonumber \\
&= -2 \mqty(E_{\Pelectron} \\ E_{\Pelectron}) \cdot \mqty(E_{\APmuon} \\ -E_{\APmuon}\cdot \cos\theta) \nonumber \\
&= -2 E_{\Pelectron}E_{\APmuon} (1+\cos\theta)
\end{align}

Then the differential cross section becomes
\begin{align}
\dv{\sigma}{\Omega} &= \frac{e^4}{32\pi^2s} \frac{4E_{\Pmuon}^2(1-\cos\theta)^2 + 4E_{\APmuon}^2(1+\cos\theta)^2}{16 E_{\Ppositron}^2} \nonumber \\
&= \frac{e^4}{128\pi^2s} \frac{E_{\Pmuon}^2(1-\cos\theta)^2 + E_{\APmuon}^2(1+\cos\theta)^2}{E_{\Ppositron}^2}
\end{align}

Now we choose the centre of momentum frame where $E_{\Pelectron} = E_{\Pmuon} = E_{\APelectron} = E_{\APmuon}$ for massless particles. Then the differential cross section becomes
\begin{equation}
\dv{\sigma}{\Omega} = \frac{e^4}{64\pi^2s} \left( 1 + \cos^2\theta \right)
\end{equation}

which can be easily integrated:
\begin{align}
\sigma &= \frac{e^4}{64\pi^2s}\int(1+\cos^2\theta) \, \dd{\Omega} \nonumber \\
&= \frac{e^4}{64\pi^2s} \left( 4\pi + 2\pi\int\limits_{-1}^1 \cos^2\theta \, \dd(\cos{\theta}) \right) \nonumber \\
&= \frac{e^4}{64\pi^2s} \left( 4\pi + \frac{4\pi}{3} \right) \nonumber \\
&= \frac{e^4}{12\pi s}.
\end{align}

Finally we use that the fine structure constant $\alpha = \frac{e^2}{4\pi}$ (see \eqref{eq:fineStructure}) to get
\begin{equation}\boxed{
\sigma = \frac{4\pi\alpha^2}{3s}.
}\end{equation}

\section{$R$ at \Pelectron \Ppositron colliders}
The process $\HepProcess{\Pelectron\Ppositron \to \Pmuon\APmuon}$ may be generalised to include other allowed particles in the final state, shown in Figure~\ref{fig:generalAnnihilation}.
\begin{figure}[th]
\centering
\unitlength = 1mm
\begin{fmffile}{generalAnnihilate}
    \begin{fmfgraph*}(65,20)
        
        \fmfleftn{i}{2}
        \fmfrightn{o}{2}
        
        \fmf{fermion,label=\Pfermion}{o1,v1,o2}
        \fmf{fermion,label=\Pe}{i2,v2,i1}
        \fmf{photon,label=\Pphoton}{v1,v2}
        
        \fmfdot{v1,v2}
        
    \end{fmfgraph*}
\end{fmffile}
\caption{The diagram for general $s$-channel \Pelectron \Ppositron annihilation with final state fermions in QED.\label{fig:generalAnnihilation}}
\end{figure}
The process is exclusively $s$-channel except in the case for final-state electrons, where $t$- and $u$- channels contribute, and in fact dominate at low energies. Also, muons are simpler to detect. For these reasons, we use the process with final-state muons as a `standard candle' and define the ratio $R$,
\begin{equation}\boxed{
R = \frac{\sigma( \Pelectron\Ppositron \to \Pquark\APquark )}{\sigma ( \Pelectron\Ppositron \to \Pmuon\APmuon )}.
}\end{equation}

At low energies, only \Pup and \Pdown quarks are accessible in the final state. They instantly hadronise into pions or $\rho$ mesons, for example. The cross section is proportional to the square of the quark charge and we simply add the cross sections for distinguishable final states,
\begin{equation}
R_{\Pup\Pdown} = \left[\left(\frac{2}{3}\right)^2 + \left(\frac{-1}{3}\right)^2 \right] \times 3 = \frac{5}{3}
\end{equation}
where the factor of 3 is due to colour multiplicity of the quark states. In this simple treatment, $R$ increases in steps to $\frac{6}{3}$, $\frac{10}{3}$, $\frac{11}{3}$ up to the CM energy $\sqrt{s} = 2m_{\Pbottom} \simeq \SI{9.3}{\giga\electronvolt}$. When the final state quarks combine to form resonances (e.g.~\Prho: \SI{770}{\mega\electronvolt}, \PJpsi: \SI{3.1}{\giga\electronvolt}, \PUpsilonFourS: \SI{10.6}{\giga\electronvolt}) $R$ increases above these predictions due to the enhanced cross section.

The Barbar \Pelectron \Ppositron collider operated at the \PUpsilonFourS resonance, allowing measurement of $R$ at about 3.84 (expected 3.67). The higher measured value can be accounted for by QCD higher order $O(\alpha_S)$ QCD corrections due to gluon radiation.

Similarly, the LEP collider operated at the $\PZ$ resonance $\sqrt{s} = \SI{91}{\giga\electronvolt}$. An analogous quantity $R_Z$ is defined
\begin{equation}
R = \frac{\sigma( \PZ \to \Pquark\APquark )}{\sigma ( \PZ \to \Pmuon\APmuon )}.
\end{equation}
To lowest order, $P_Z = 20.09$ and it is measured to be $20.79 \pm 0.04$. The $3.5\%$ discrepancy is fully explained by higher order QCD corrections involving gluons.
\begin{figure}[hb]
\centering
\unitlength = 1mm
\begin{fmffile}{zgluonrad}
    \begin{fmfgraph*}(65,20)
        
        \fmfleftn{i}{1}
        \fmfrightn{o}{4}
        
        \fmf{zigzag,label=\PZ,tension=2}{i1,v1}
        \fmfstraight
        \fmf{fermion, label=\Pquark}{o1,v1,v2,o4}
        
        \fmffreeze
        \fmf{gluon,tension=0}{v2,o2}
        
        \fmfdot{v1}
        
    \end{fmfgraph*}
\end{fmffile}
\caption{$\PZ$ decay into quark final states with gluon radiation. This diagram contributes to the $O(\alpha_S)$ QCD correction to $R_Z$.\label{fig:Zgluonrad}}
\end{figure}

\section{Helicity conservation at high energies}
Recall the chiral projection operators from Section \ref{sec:gammaMatrix},
\begin{equation*}
P_R = \frac{1}{2}\left( 1 + \gamma^5 \right), \quad P_L = \frac{1}{2}\left( 1 - \gamma^5 \right).
\end{equation*}
As discussed above, in the limit where $m \rightarrow 0$, helicity and chirality have the same eigenstates. The left-handed chiral state is given, for example, by $U_L = P_L U$.

Using the fact that $\gamma^5$ -- and hence $P_L$ and $P_R$ -- is Hermitian and $\gamma^\mu \gamma^5 = -\gamma^5 \gamma^\mu$, we see that $P_L\gamma^\mu = \gamma^\mu P_R$ and therefore
\begin{equation}
\overline{U}_L = (P_L U)^\dagger \gamma^0 = U^\dagger P_L \gamma^0 = U^\dagger \gamma_0 P_R = \overline{U} P_R.
\end{equation}
Similarly, $\overline{U}_R = \overline{U}P_L$.

Now consider the QED current
\begin{equation}
\overline{U}\gamma^\mu{U} = (\overline{U}_L + \overline{U}_R)\gamma^\mu(U_L + U_R)
\end{equation}
and look at a cross-term specifically,
\begin{align}
\overline{U}_L \gamma^\mu U_R &= \overline{U} P_R \gamma^\mu P_R U \nonumber \\
&= \overline{U} \gamma^\mu P_L P_R U \nonumber \\
&= 0\end{align}
because $P_LP_R=0$. This means there are no chirality-changing currents in QED and hence QED conserved chirality, or equivalently helicity in the high energy limit.
