\chapter{Relativistic Spin-$\frac{1}{2}$ particles}
\section{Spin in non-relativistic QM}
The quantum states corresponding to spin-$\frac{1}{2}$ particles are eigenstates of the spin operator,
\begin{equation}
\hat{S}_i = \frac{1}{2}\sigma_i
\end{equation}
where $\sigma_i$ are the Pauli matrices,
\begin{equation}
\sigma_x = \mqty(0 & 1 \\ 1 & 0), \quad
\sigma_y = \mqty(0 & -i \\ i & 0), \quad
\sigma_z = \mqty(1 & 0 \\ 0 & -1).
\end{equation}
Application of $\hat{S}_z$ on a state of definite spin projection on the z-axis gives
\begin{equation}
\hat{S}_z \ket{s, m} = m \ket{s, m}
\end{equation}
and application of the operator $\hat{S}^2 = \sum_i \hat{S}_i^2$ -- which corresponds to the total spin angular momentum of the state -- gives
\begin{equation}
\hat{S}^2 \ket{s, m} = s(s+1) \ket{s, m}.
\end{equation}

The commutator of any two spin operators is
\begin{equation}
\left[ \hat{S}_i, \hat{S}_j \right] = i \epsilon_{ijk} \hat{S}_k
\end{equation}
and they have the anticommutation relation
\begin{equation}
\left\{ \hat{S}_i, \hat{S}_j \right\} = \delta_{ij}.
\end{equation}
Therefore, we have that
\begin{align}
\sigma_i \sigma_j &= \frac{[ \sigma_i, \sigma_j ] + \{ \sigma_i, \sigma_j \}}{2}  \nonumber \\
&= \delta_{ij} + i\epsilon_{ijk}\sigma_k.
\end{align}
So for any two vectors, $\vec{a}$ and $\vec{b}$,
\begin{align}
(\vec{\sigma}\cdot\vec{a})(\vec\sigma\cdot\vec{b}) &= \sigma_i \, a_i \, \sigma_j \, b_j \nonumber \\
&= \sigma_i \, \sigma_j \, (a_i \, b_j) \nonumber \\
&= (\delta_{ij} + i\epsilon_{ijk}\sigma_k)(a_i \, b_j) \nonumber \\
&= a_i \, b_i + i \, \sigma_k \, \epsilon_{kij} \, a_i \, b_j \nonumber \\
&= \vec{a}\cdot\vec{b} + i \vec{\sigma}\cdot(\vec{a}\times\vec{b}).
\end{align}
Applying this result to $\vec{a} = \vec{b} = \vec{p}$, the momentum operator,
\begin{align}
(\vec{\sigma}\cdot\vec{p})(\vec{\sigma}\cdot\vec{p}) &= \vec{p}\cdot\vec{p} + i \vec{\sigma}\cdot\underbrace{(\vec{p} \times \vec{p})}_0 \nonumber \\
&= \abs{\vec{p}}^2. \label{eq:sigmaDotP2}
\end{align}
This allows us to express the energy as
\begin{equation}
E = \frac{(\vec{\sigma}\cdot\vec{p})(\vec{\sigma}\cdot\vec{p})}{2m} + V
\end{equation}
which leads to the correct energy for a particle with a gyromagnetic ratio if the EM coupling is included.

\section{The Dirac equation}
To avoid the negative energy solutions seen in the treatment of the Klein-Gordon equation, Dirac proposed an equation of motion that is linear in $\pdv{t}$. Also, there should be distinct solutions for particles with opposite spin.

\begin{equation}\boxed{
\hat{H}\psi = \left( \vec{\alpha}\cdot{\vec{p}} + \beta m \right) \psi
}\end{equation}
is the Dirac equation. Here, $\alpha_i$ and $\beta$ are $4\times4$ matrices and $\psi$ therefore a 4-component Dirac spinor.

\subsection{Constraints on $\alpha_i$, $\beta$}
For the Dirac equation to describe relativistic particles, it must agree with the relation
\begin{equation}
E^2 = p^2 + 2m.
\end{equation}
Expanding out the operators in the Dirac equation above,
\begin{align}
\hat{H}^2 &= \left( \alpha_i p_i + \beta m \right)\left( \alpha_j p_j + \beta m \right) \nonumber \\
&= \alpha_i \alpha_j p_i p_j + \left\{ \alpha_i, \beta \right\}p_i + \beta^2 m^2.
\end{align}
So we see that the matrices $\alpha_i$ and $\beta$ have the constraints:
\begin{itemize}
\item $\beta^2 = I_4$
\item $\alpha_i \alpha_j = \delta_{ij} I_4$
\item $\left\{ \alpha_i, \beta \right\} = 0 \quad \Rightarrow \quad \alpha_i \beta = - \beta \alpha_i$
\end{itemize}
This final constraint further requires that $\alpha_i$ and $\beta$ are all traceless:
\begin{align*}
\tr(\alpha_i \beta) &= -\tr(\beta \alpha_i) \\
\tr(\alpha_i \beta^2) &= -\tr(\beta \alpha_i \beta) \\
\tr(\alpha_i) &= -\tr(\beta^2 \alpha_i) \quad \text{by cyclicity of the trace} \\
&= -\tr(\alpha_i) = 0.
\end{align*}
Moreover, we require that $\alpha_i$ and $\beta$ are Hermitian.

These constraints still allow for many different representations of the Dirac equation. A commonly used choice is
\begin{equation}
\alpha_i = \mqty(\admat[0]{\sigma_i, \sigma_i}), \quad \beta = \mqty(\dmat[0]{I_2, -I_2}).
\end{equation}

\subsection{Covariant form}
Using the Schr{\"o}dinger equation, express the Dirac equation in terms of differential operators
\begin{equation}
i \pdv{\psi}{t} = -i \vec{\alpha}\cdot\vec{\nabla}\psi + \beta m \psi
\end{equation}
and pre-multiply by $\beta$,
\begin{equation}
i \beta \pdv{\psi}{t} = -i \beta \vec{\alpha}\cdot\psi + \beta^2 m \psi.
\end{equation}
Now we recognise that $\beta^2=I_4$ and define the Dirac gamma matrices $\gamma^0 = \beta$, $\gamma^k = \beta\alpha^k$. This gives
\begin{equation}\boxed{
\left(i \gamma^\mu \partial_\mu - m \right)\psi = 0 \label{eq:covDirac}
}.\end{equation}
This is the covariant form of the Dirac equation. Explicitly, the Dirac matrices are
\begin{align*}
\gamma^0 = \mqty(\dmat[0]{I_2,-I_2}), \quad \gamma^k = \mqty(\admat[0]{\sigma^k, -\sigma^k}).
\end{align*}

\subsection{Properties of the $\gamma$ matrices}
Clearly $\gamma^0$ is Hermitian,
\begin{equation}
(\gamma^0)^\dagger = \gamma^0
\end{equation}
but the $\gamma^k$ are anti-Hermitian,
\begin{align}
(\gamma^k)^\dagger &= (\beta\alpha^k)^\dagger \nonumber \\
&= (\alpha^k)^\dagger \beta^\dagger \nonumber \\
&= \alpha^k \beta \nonumber \\
&= -\beta \alpha^k \nonumber \\
&= -\gamma^k.
\end{align}
Also we have that
\begin{equation}
(\gamma^0)^2 = I_4
\end{equation}
and
\begin{equation}
(\gamma^k)^2 = -I_4.
\end{equation}
Finally, the anticommutator
\begin{equation}
\left\{ \gamma^0, \, \gamma^k \right\} = 0.
\end{equation}

These properties will prove useful in the following manipulations of the Dirac equation.

\section{Adjoint Dirac equation}
Taking the Hermitian conjugate of \eqref{eq:covDirac} and expanding out the space and time components,
\begin{equation}
-i \partial_0 {\psi^\dagger} \gamma^0 + i \partial_k {\psi^\dagger} \gamma^k - m \psi^\dagger = 0.
\end{equation}
Now multiply by $-\gamma^0$ from the right and use $\gamma^0\gamma^k = -\gamma^k\gamma^0$,
\begin{equation}
i \partial_0 {\psi^\dagger} \gamma^0 \gamma^0 + i \partial_k \psi^\dagger \gamma^0 \gamma^k + m\psi^\dagger \gamma^0 = 0.
\end{equation}
Here is it natural to identify the adjoint Dirac spinor,
\begin{equation}\boxed{
\overline{\psi} \equiv \psi^\dagger \gamma^0
}\end{equation}
so the adjoint Dirac equation becomes
\begin{equation}\boxed{
i\left(\partial_\mu \overline{\psi}\right) \gamma^\mu + m \overline{\psi} = 0 \label{eq:adjDirac}
}\end{equation}

\subsection{Conserved current}
Pre-multiply the standard Dirac equation, \eqref{eq:covDirac}, by the adjoint spinor
\begin{equation}
i \overline{\psi} \gamma^\mu \left(\partial_\mu \psi\right) - m \overline{\psi}\psi = 0
\end{equation}
and post-mulitply the adjoint Dirac equation, \eqref{eq:adjDirac}, by $\psi$,
\begin{equation}
i\left(\partial_\mu\overline{\psi}\right)\gamma^\mu \psi + m\overline{\psi}\psi = 0.
\end{equation}
Adding these, we see that
\begin{equation}
\overline{\psi} \gamma^\mu \left(\partial_\mu \psi\right) + \left(\partial_\mu\overline{\psi}\right)\gamma^\mu \psi = \partial_\mu\left( \overline{\psi} \gamma^\mu \psi \right) = 0
\end{equation}
where we can identify the 4-current density,
\begin{equation}
j^\mu = \overline{\psi} \gamma^\mu \psi.
\end{equation}
Now the probability density is given by
\begin{align*}
j^0 &= \overline{\psi}\gamma^0\psi \\
&= \psi^\dagger \gamma^0 \gamma^0 \psi \\
&= \abs{\psi}^2
\end{align*}
which is always positive. Hence the Dirac equation avoids the negative probabilities we saw in the treatment of the Klein-Gordon equation.

As usual the electromagnetic charge-current density is given by $-e\overline{\psi}\gamma^\mu\psi$ for electrons.

\section{Free particle solutions}
Consider solutions of the form
\begin{equation}
\psi = u(p) \, e^{-ipx}
\end{equation}
where $u(p)$ is a 4-component spinor and $p$ and $x$ are understood to be 4-vectors.

Substituting this into the covariant form of the Dirac equation, \eqref{eq:covDirac},
\begin{align}
0 &= \left( i \gamma^\mu \partial_\mu - m \right) u(p) \, e^{-ipx} \nonumber \\
&= \left( \gamma^\mu p_\mu - m \right) u(p) \, e^{-ipx}.
\end{align}

Now define the Feynman slash notation, $\fsl{p} \equiv \gamma^\mu p_\mu$. Solutions of the above equation are given by spinors $u(p)$ that satisfy
\begin{equation}
(\fsl{p} - m) u(p) = 0.
\end{equation}

To find these solutions, it is helpful to return to the original $\alpha_i, \beta$ matrices. Do this by expanding out the $\fsl{p}$ and pre-multiplying by $\gamma^0$:
\begin{align}
\gamma^0(\fsl{p}-m) &= \gamma^0 \gamma^0 E - \gamma^0 \gamma^k p_k - \gamma^0 m \nonumber \\
&= E - \alpha^k p^k - \beta m
\end{align}
so we look for solutions of
\begin{equation}
\left( E - \alpha^k p_k - \beta m \right) u(p) = 0.
\end{equation}

\subsection{Particles at rest}
For a particle at rest $p_k=0$ so we have to solve
\begin{equation}
\left(E - \beta m \right) u(p) = 0.
\end{equation}
Solutions exist for
\begin{equation}
\mdet{\dmat[0]{(E-m)I_2, -(E+m)I_2}} = 0
\end{equation}
which has energy solutions $E = \pm m$. We have not escaped the negative energy solutions after all! In fact, this leads to four distinct solutions for the spinor
\begin{equation}
u_{1,2} = \mqty(\chi_\pm \\ 0) , \quad u_{3,4} = \mqty(0 \\ \chi_\pm) \quad \text{where} \quad \chi_+ = \mqty(1 \\ 0), \quad \chi_- = \mqty(0 \\ 1). \label{eq:solutionsAtRestDirac}
\end{equation}
It is simple to see that $u_{1,2}$ are associated with positive energy states and $u_{3,4}$ with negative energy states. Furthermore, the spinors $\chi_\pm$ correspond to particles with spin projections $\pm \frac{1}{2}$.

\subsection{Moving particle solutions}
Now consider the case where $p_k \neq 0$. This gives the eigenvalue problem
\begin{equation}
\left[ \mqty(\admat[0]{\vec{\sigma}, \vec{\sigma}}) \cdot \vec{p} + \mqty(\dmat[0]{I_2, I_2}m) \right] \mqty(u_A \\ u_B) = E \mqty(u_A \\ u_B)
\end{equation}
or the simultaneous equations
\begin{align}
u_A &= \frac{\vec{\sigma}\cdot\vec{p}}{E-m} \, u_B \\
u_B &= \frac{\vec{\sigma}\cdot\vec{p}}{E+m} \, u_A.
\end{align}
There are still infinitely many solutions but $u_B$ is determined from $u_A$ or vice-versa. To agree with our choices for the spinor at rest, \eqref{eq:solutionsAtRestDirac}, we choose that for positive energy solutions
\begin{equation}
u_{1,2}(p) = \mathcal{N} \mqty(\chi_\pm \\ \frac{\vec{\sigma}\cdot\vec{p}}{E+m} \chi_\pm)
\end{equation}
and for negative energies, $E = -\abs{E} < 0$,
\begin{equation}
u_{1,2}(p) = \mathcal{N} \mqty(-\frac{\vec{\sigma}\cdot\vec{p}}{\abs{E}+m} \chi_\pm \\ \chi_\pm)
\end{equation}

\subsection{Antiparticles}
So far we have seen that there are four distinct solutions to the Dirac equation, accounting for each spin and positive and negative energy states. In keeping with the Feynman-Stueckelberg interpretation, we ought to demonstrate that the negative energy solutions of the Dirac equation correspond to the positron.

The full spinor states for the negative energy solutions are
\begin{align}
\psi &= u_{3,4}(p) \, e^{-ipx} \nonumber \\
&= u_{3,4}(p) \, \exp[-i(Et - \vec{p}\cdot\vec{x})] \nonumber \\
&= u_{3,4}(p) \, \exp[i(\abs{E}t + \vec{p}\cdot\vec{x})]
\end{align}
so again the negative energy solution can be interpreted as a positive energy particle travelling backwards in time and space. Hence, we can write the spinor as
\begin{equation}
\phi = v_{1,2}(p) e^{-ipx}
\end{equation}
where $v_{1,2}(p) = u_{4,3}(-p)$ and this describes a positive energy particle travelling forwards through time. Note that since what we have done is essentially a $CT$ (or equivalently $P$) transformation, the spin states are reversed such that
\begin{equation}
v_1(p) = \mathcal{N} \mqty(\frac{\vec{\sigma}\cdot\vec{p}}{E+m} \chi_- \\ \chi_-)
\end{equation}
corresponds to a spin \emph{up} positron, and
\begin{equation}
v_2(p) = \mathcal{N} \mqty(\frac{\vec{\sigma}\cdot\vec{p}}{E+m} \chi_+ \\ \chi_+)
\end{equation}
corresponds to spin down.

\section{Orthogonality and normalisation}
Without loss of generality, consider electrons travelling along the $z$-axis. Then for a spin-up electron,
\begin{equation}
u_1(p) = \mathcal{N} \mqty(1 \\ 0 \\ \frac{p}{E+m} \\ 0)
\end{equation}
and for spin-down
\begin{equation}
u_2(p) = \mathcal{N} \mqty(0 \\ 1 \\ 0 \\ \frac{-p}{E+m})
\end{equation}

Consider two electron wavefunctions, $\psi_1$, $\psi_2$, with opposite spin. Their overlap is given by
\begin{equation}
\int \psi_1^\dagger \psi_2 \, \dd[3]{\vec{x}}.
\end{equation}
Now
\begin{equation}
\psi_1^\dagger \psi_2 = \mathcal{N}^* \mathcal{N} \, \mqty(1, \, 0, \, \frac{p}{E+m}, \, 0) \mqty(0 \\ 1 \\ 0 \\ \frac{-p}{E+m}) = 0
\end{equation}
and therefore the overlap is zero. This shows that the solutions to the Dirac equation are orthogonal.

The normalisation condition is given by
\begin{equation}
2E = \int_V \psi_1^\dagger \psi_1 \, \dd[3]{\vec{x}}.
\end{equation}
For a spin-up electron travelling along the $z$-axis,
\begin{align}
\psi_1^\dagger \psi_1 &= \abs{\mathcal{N}}^2 \, \mqty(1, \, 0, \, \frac{p}{E+m}, \, 0) \mqty(1 \\ 0 \\ \frac{p}{E+m} \\ 0) \nonumber \\
&= \abs{\mathcal{N}}^2 \, \left( 1+\frac{p^2}{(E+m)^2} \right) \nonumber \\
&= \abs{\mathcal{N}}^2 \, \left( 1+\frac{E-m}{E+m} \right) \nonumber \\
&= \abs{\mathcal{N}}^2 \, \frac{2E}{E+m}
\end{align}
Therefore the normalisation condition gives
\begin{equation}
2E = \mathcal{N}^2 \, \frac{2E}{E+m} \, V
\end{equation}
\begin{equation}
\Rightarrow \quad \mathcal{N} = \sqrt{\frac{E+m}{V}}
\end{equation}
Often we will use $V=1$ for convenience.

\section{Helicity}
We would like a way to describe the spin of a Dirac particle that is consistent in time. Such a good quantum number must commute with the Dirac Hamiltonian, $\hat{H}$, which the usual QM operator $\hat{S}$ does not. To find a suitable operator, it is instructive to expand out the Hamiltonian. According to the Dirac equation,
\begin{align}
\hat{H} \mqty(u_A \\ u_B) &= \left( \vec{\alpha}\cdot\vec{p} + \beta m \right) \mqty(u_A \\ u_B) \nonumber \\
&= \mqty(m & \vec{\sigma}\cdot\vec{p} \\ \vec{\sigma}\cdot\vec{p} & -m)\mqty(u_A \\ u_B)
\end{align}
where it is to be understood that $m$ stands for $\smqty(m & 0 \\ 0 & m)$. We see from the matrix form of the Dirac Hamiltonian that $\vec{\sigma}\cdot\vec{p}$ commutes with $\hat{H}$ and hence there exists a corresponding conserved quantity.

\subsection{Helicity}
Define the helicity operator
\begin{equation}\boxed{
h \equiv \frac{1}{2} \vec{\sigma}\cdot\hat{\vec{p}}
}\end{equation}
where $\hat{\vec{p}} = \vec{p}/\abs{\vec{p}}$. Notice that $\vec{\sigma}/2 = \vec{S}$ so that $h = \vec{S}\cdot\hat{\vec{p}}$ is nothing but the projection of the particle's spin along its direction of travel.

To find the spectrum of helicities, we solve the eigenvalue problem for the matrix $\frac{1}{2} \vec{\sigma}\cdot\hat{\vec{p}}$. This is most easily done by writing out the momentum in polar coordinates:
\begin{equation}
\hat{\vec{p}} = \mqty(\sin\theta\cos\phi \\ \sin\theta\sin\phi \\ \cos\theta)
\end{equation}
then
\begin{equation}
\vec{\sigma}\cdot\hat{\vec{p}} = \mqty(\cos\theta & \sin\theta \, e^{-i\phi} \\
\sin\theta \, e^{i\phi} & -\cos\theta).
\end{equation}
This gives the eigenvalue problem,
\begin{equation}
\frac{1}{2} \mqty(\cos\theta & \sin\theta \, e^{-i\phi} \\
\sin\theta \, e^{i\phi} & -\cos\theta) \mqty(u_A \\ u_B) = \lambda \mqty(u_A \\ u_B)
\end{equation}
which has the characteristic equation
\begin{equation}
-\left( \cos\theta - 2\lambda \right) \left( \cos\theta + 2\lambda \right) - \sin^2\theta = 0
\end{equation}
with solutions
\begin{equation}
\lambda_\pm = \pm\frac{1}{2}.
\end{equation}

\section{The $\gamma^5$ matrix}\label{sec:gammaMatrix}
Define the useful $\gamma^5$ matrix
\begin{equation}
\gamma^5 = i \gamma^0 \gamma^1 \gamma^2 \gamma^3.
\end{equation}
In the Dirac-Pauli representation, it takes the form
\begin{equation}
\gamma^5 = \mqty(0 & I_2 \\ I_2 & 0)
\end{equation}
with the properties
\begin{align}
(\gamma^5)^\dagger &= \gamma^5 \\
(\gamma^5)^2 &= I_4 \\
\gamma^5 \gamma^\mu &= -\gamma^\mu \gamma^5.
\end{align}

Now consider the action of $\gamma^5$ on a solution of the Dirac equation,
\begin{equation}
\gamma^5 \mqty(u_A \\ u_B) = \mqty(0 & I_2 \\ I_1 & 0) \mqty(\chi \\ \frac{\vec{\sigma}\cdot\vec{p}}{E+m} \chi) = \mqty(\frac{\vec{\sigma}\cdot\vec{p}}{E+m} \chi \\ \chi).
\end{equation}
For ultra-relativistic particles $m \rightarrow 0$ and hence $\frac{\vec{\sigma}\cdot\vec{p}}{E+m} \rightarrow \vec{\sigma}\cdot\hat{\vec{p}}$. We also use the fact that $(\vec{\sigma}\cdot\hat{\vec{p}})(\vec{\sigma}\cdot\hat{\vec{p}}) = I_2$, as shown in \eqref{eq:sigmaDotP2}.
\begin{align}
\gamma^5 \mqty(u_A \\ u_B) &= \mqty(\left(\vec{\sigma}\cdot\hat{\vec{p}}\right) \chi \\ \chi) \nonumber \\
&= \mqty(\left(\vec{\sigma}\cdot\hat{\vec{p}}\right) \chi \\ \left(\vec{\sigma}\cdot\hat{\vec{p}}\right)^2 \chi) \nonumber \\
&= \vec{\sigma}\cdot\hat{\vec{p}} \, \mqty(\chi \\ \left(\vec{\sigma}\cdot\hat{\vec{p}}\right)\chi) \nonumber \\
&= \vec{\sigma}\cdot\hat{\vec{p}} \, \mqty(u_A \\ u_B).
\end{align}
Therefore, in the limit where $m \rightarrow 0$, the helicity operator becomes $h \rightarrow {\gamma^5/2}$.

Define the chiral projection operators,
\begin{equation}
P_R = \frac{1}{2}\left( 1 + \gamma^5 \right), \quad P_L = \frac{1}{2}\left( 1 - \gamma^5 \right).
\end{equation}
In the limit where $m \rightarrow 0$, a helicity eigenstate corresponds to a chiral state -- i.e.~$h=1$ is a right-handed state. For small $m$, a helicity eigenstate will be mostly one chirality plus a small amount of the other.

\section{Completeness relations}
The completeness relations are used extensively in the evaluation of Feynman diagrams.

Sum over the spin states,
\begin{align}
\sum_{s=\pm} u_s(p) \, \overline{u}_s(p) &= \sum_{s=\pm} (E+m) \, \mqty(\chi_s \\ \frac{\vec{\sigma}\cdot\vec{p}}{E+m} \chi_s) \mqty(\chi_s^\dagger, \, -\frac{\left(\vec{\sigma}\cdot\vec{p}\right)^\dagger}{E+m} \chi_s^\dagger) \nonumber \\
&= \sum_{i=\pm} (E+m) \, \mqty(\chi_s \chi_s^\dagger & -\frac{\left(\vec{\sigma}\cdot\vec{p}\right)^\dagger}{E+m} \chi_s \chi_s^\dagger \\ \frac{\vec{\sigma}\cdot\vec{p}}{E+m} \chi_s \chi_s^\dagger & -\frac{E-m}{E+m} \chi_s \chi_s^\dagger)
\end{align}
where for the bottom-right element we have used $(\vec{\sigma}\cdot\vec{p})^2 = \abs{p}^2 = E^2 - m^2 = (E-m)(E+m)$.

The sum over spin states can be easily performed:
\begin{equation}
\sum_{i=\pm} \chi_s \chi_s^\dagger = \mqty(1 \\ 0) \mqty(1,\, 0) + \mqty(0 \\ 1) \mqty(0,\, 1) = \mqty(\dmat[0]{1,1}) = I_2
\end{equation}
So we have that
\begin{align}
\sum_{i=\pm} u_s(p) \, \overline{u}_s(p) &= (E+m) \, \mqty(I_2 & -\frac{\vec{\sigma}\cdot\vec{p}}{E+m} \\ \frac{\vec{\sigma}\cdot\vec{p}}{E+m} & \frac{m-E}{E+m}) \nonumber \\
&= \mqty(E+m & -\vec{\sigma}\cdot\vec{p} \\ \vec{\sigma}\cdot\vec{p} & m-E) \label{eq:spinSum}
\end{align}

Now consider the matrix given by
\begin{align}
\fsl{p} + m &= \gamma^\mu p_\mu + m \\
&= \mqty(E & 0 \\ 0 & -E) - \mqty(0 & \vec{\sigma}\cdot\vec{p} \\ -\vec{\sigma}\cdot\vec{p} & 0) + \mqty(\dmat[0]{m,m}) \nonumber \\
&= \mqty(E+m & -\vec{\sigma}\cdot\vec{p} \\ \vec{\sigma}\cdot\vec{p} & m-E) \label{eq:pslash+m}
\end{align}

Comparing \eqref{eq:spinSum} and \eqref{eq:pslash+m} gives the completeness relation
\begin{equation}\boxed{
\sum_{i=\pm} u_s(p) \, \overline{u}_s(p) = \fsl{p} + m \label{eq:completeness}
}.\end{equation}

Similarly for antiparticles,
\begin{equation}
\sum_{i=\pm} v_s(p) \, \overline{v}_s(p) = \fsl{p} - m.
\end{equation}

\section{Forms of interaction in Dirac theory}
\textbf{Scalar} (even parity): $\overline{\psi} \psi$

\textbf{Pseudoscalar} (odd parity): $\overline{\psi} \gamma^5 \psi$

\textbf{Polar vector} (odd parity): $\overline{\psi} \gamma^\mu \psi$. Fermi $\beta$ decay is described by a polar vector current.

\textbf{Axial vector} (even parity): $\overline{\psi}\gamma^5 \gamma^\mu \psi$. The weak interaction is a mixture of polar and axial vector interactions.

\textbf{Tensor}: $\overline{\psi}\sigma^{\mu\nu} \psi$. This type of interaction has not been observed by experiment, although does provide a description of the mechanism behind the anomalous magnetic moment. Spin-2 gravitons would have interactions of this form.

\section{Trace theorems}\label{sec:Trace}
In scattering problems, we will often see calculations involving the traces of $\gamma$-matrices. What follows will simplify these calculations with some algebraic results.

We start with the facts that traces are associative:
\begin{equation}
\Tr(A+B) \equiv \Tr(A) + \Tr(B)
\end{equation}
and are unchanged by cyclic permutations of the argument:
\begin{equation}
\Tr(AB \ldots YZ) \equiv \Tr(ZAB \ldots Y).
\end{equation}

Also the algebra of the $\gamma$-matrices is defined by their anticommutation relation,
\begin{equation}
\gamma^\mu \gamma^\nu + \gamma^\nu \gamma^\mu = 2g^{\mu\nu} I_4 \label{eq:gammaAnticommute}
\end{equation}
where the presence of the identity has been made explicit. Taking the trace,
\begin{equation}
\Tr(\gamma^\mu \gamma^\nu) + \Tr(\gamma^\nu \gamma^\mu) = 2g^{\mu\nu} \Tr(I_4)
\end{equation}
and using the cyclicity of the trace,
\begin{equation}\boxed{
\Tr(\gamma^\mu \gamma^\nu) = 4g^{\mu\nu} \label{eq:trace2}
}.\end{equation}

The trace of any odd number of $\gamma$-matrices can be shown to be zero by inserting $\gamma^5 \gamma^5 = I_4$ into the trace. For example,
\begin{align}
\Tr(\gamma^\mu \gamma^\nu \gamma^\rho) &= \Tr(\gamma^5 \gamma^5 \gamma^\mu \gamma^\nu \gamma^\rho) \nonumber \\
&= \Tr( \gamma^5 \gamma^\mu \gamma^\nu \gamma^\rho \gamma^5) \nonumber \\
&= -\Tr(\gamma^5 \gamma^5 \gamma^\mu \gamma^\nu \gamma^\rho)
\end{align}
where the last line is reached by commuting the last $\gamma^5$ through three $\gamma$-matrices, each time introducing a factor $-1$. Hence we have that $\Tr(\gamma^\mu \gamma^\nu \gamma^\rho) = -\Tr(\gamma^\mu \gamma^\nu \gamma^\rho)$ which can only be true for
\begin{equation}\boxed{
\Tr(\gamma^\mu \gamma^\nu \gamma^\rho) = 0
}.\end{equation}
This is easily extended to the case for any odd number of $\gamma$-matrices.

The trace of four $\gamma$-matrices can be obtained using \eqref{eq:gammaAnticommute}:
\begin{align}
\gamma^\mu \gamma^\nu \gamma^\rho \gamma^\sigma &= \left( 2g^{\mu\nu} - \gamma^\nu \gamma^\mu \right) \gamma^\rho \gamma^\sigma \nonumber \\
&= 2g^{\mu\nu}\gamma^\rho\gamma^\sigma - 2g^{\mu\rho}\gamma^\nu\gamma^\sigma + \gamma^\nu\gamma^\rho\gamma^\mu\gamma^\sigma \nonumber \\
&= 2g^{\mu\nu}\gamma^\rho\gamma^\sigma - 2g^{\mu\rho}\gamma^\nu\gamma^\sigma + 2g^{\mu\sigma}\gamma^\nu\gamma^\rho - \gamma^\nu\gamma^\rho\gamma^\sigma\gamma^\mu
\end{align}
Hence,
\begin{equation}
\Tr(\gamma^\mu \gamma^\nu \gamma^\rho \gamma^\sigma) = g^{\mu\nu}\Tr(\gamma^\rho\gamma^\sigma) - g^{\mu\rho}\Tr(\gamma^\nu\gamma^\sigma) + g^{\mu\sigma}\Tr(\gamma^\nu \gamma^\rho)
\end{equation}
Evaluating the traces with \eqref{eq:trace2} gives the result
\begin{equation}\boxed{
\Tr(\gamma^\mu \gamma^\nu \gamma^\rho \gamma^\sigma) = 4g^{\mu\nu}g^{\rho\sigma} - 4g^{\mu\rho}g^{\nu\sigma} + 4g^{\mu\sigma}g^{\nu\rho}
}\end{equation}

The full set of trace theorems is:
\begin{itemize}
\item $\Tr(I_4) = 4$;
\item the trace of any odd number of $\gamma$-matrices is zero;
\item $\Tr(\gamma^\mu \gamma^\nu) = 4g^{\mu\nu}$;
\item $\Tr(\gamma^\mu \gamma^\nu \gamma^\rho \gamma^\sigma) = 4g^{\mu\nu}g^{\rho\sigma} - 4g^{\mu\rho}g^{\nu\sigma} + 4g^{\mu\sigma}g^{\nu\rho}$;
\item the trace of $\gamma^5$ multiplied by an odd number of $\gamma$-matrices is zero;
\item $\Tr(\gamma^5) = 0$;
\item $\Tr(\gamma^5 \gamma^\mu \gamma^\nu) = 0$; and
\item $\Tr(\gamma^5 \gamma^\mu \gamma^\nu \gamma^\rho \gamma^\sigma) = 4i \varepsilon^{\mu\nu\rho\sigma}$, where $\varepsilon^{\mu\nu\rho\sigma}$ is the antisymmetric tensor under the exchange of any two indices.
\end{itemize}
