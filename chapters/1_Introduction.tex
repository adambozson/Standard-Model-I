\chapter{Introduction: Matter and Forces}
The definition of the Standard Model (SM) is not fully agreed upon in the scientific community. The minimal acceptable definition is that of the unified electroweak (EW) interaction according to Glashow, Salam, and Weinberg (GSW). Quantum chromodynamics (QCD) explains the strong force, but there is currently no theory of unified EW and QCD interactions. The most commonly used Standard Model is that of EW plus QCD interactions. In this model, neutrinos are massless. However, in principle, the SM could include the phenomena of neutrino oscillations by adding a kinetic term to the Standard Model Lagrangian, $\mathcal{L}_{SM}$. If neutrinos were found to be Majorana fermions (their own antiparticles), this would explain the small neutrino mass. The widest definition of the Standard Model includes general relativity and cosmology. This extreme version is not generally used, since its large- and small-scale behaviours cannot presently be unified. In what follows, we work with SM = EW + QCD.

\section{Matter content of the universe}
The Standard Model splits the fundamental particles into fermions (spin-$\frac{1}{2}$) and bosons (integer spin). Fermions are the building-blocks of matter due to the Pauli exclusion principle. The fundamental bosons are the results of gauge invariances and serve to mediate the forces of the Standard Model.

Fermions are further divided into leptons and quarks.

\subsection{Leptons}
\begin{table}[h]
\centering
\begin{tabular}{cccc}
\toprule
I & II & III\\
\midrule
$\begin{pmatrix}\Pnue \\ \Pe \end{pmatrix}$ &
$\begin{pmatrix}\Pnum \\ \Pmu \end{pmatrix}$ &
$\begin{pmatrix}\Pnut \\ \Ptau \end{pmatrix}$ \\
\bottomrule
\end{tabular}
\caption{The leptons of the Standard Model arranged into three generations. The top row has charge $Q=0$ and the bottom row $Q=-1$.\label{tab:leptons}}
\end{table}

\begin{table}
\centering
\begin{tabular}{cc}
\toprule
\Plepton & Mass \\
\midrule
\Pe & \SI{511}{\kilo\electronvolt\per c^2} \\
\Pmu & \SI{106}{\mega\electronvolt\per c^2} \\
\Ptau & \SI{1.78}{\giga\electronvolt\per c^2}\\
\bottomrule
\end{tabular}
\caption{The masses of the charged leptons.\label{tab:leptonMass}}
\end{table}

Leptons do not experience the strong force (they are color neutral) and do not mix. Due to its mass, the \Ptau lepton can decay hadronically, whereas muon decay is purely leptonic. The most common muon decay is \HepProcess{\Pmuon\to\Pelectron\Pnum\APnue}.

\subsection{Quarks}
Similarly, the three generations of quarks are organised into isospin doublets.
\begin{table}[h]
\centering
\begin{tabular}{ccc}
\toprule
I & II & II \\
\midrule
$\begin{pmatrix}\Pup \\ \Pdown \end{pmatrix}$ &
$\begin{pmatrix}\Pcharm \\ \Pstrange \end{pmatrix}$ &
$\begin{pmatrix}\Ptop \\ \Pbottom \end{pmatrix}$ \\
\bottomrule
\end{tabular}
\caption{The quarks arranged into three isospin doublets. The top row has its third component of isospin $I_3=+\frac{1}{2}$ and the bottom $I_3=-\frac{1}{2}$. Equivalently, they have charges $Q=+\frac{2}{3}$ and $Q=-\frac{1}{3}$, respectively.\label{tab:quarks}}
\end{table}

It is difficult to assign masses to the individual quarks since they are always bound. Most of the proton's mass is created dynamically by the strong interaction between the light valence quarks. For heavy quarks, assigning a mass from the difference between hadron masses becomes easier, and approximate masses are given in Table \ref{tab:quarkMass}.

\begin{table}[h]
\centering
\begin{tabular}{cc}
\toprule
\Pquark & Mass \\
\midrule
u & \SI{2.2}{\mega\electronvolt\per c^2}\\
d & \SI{5}{\mega\electronvolt\per c^2}\\
c & \SI{1.25}{\giga\electronvolt\per c^2}\\
s & \SI{95}{\mega\electronvolt\per c^2}\\
t & \SI{174.2}{\giga\electronvolt\per c^2}\\
b & \SI{4.2}{\giga\electronvolt\per c^2}\\
\bottomrule
\end{tabular}

\caption{Approximate quark masses. It is difficult to assign masses to the lighter quarks due to quark confinement and their strong interactions (see main body).\label{tab:quarkMass}}.
\end{table}

\subsection{Bosons}
The fundamental Standard Model bosons and their properties are listed in Table \ref{tab:boson}.
\begin{table}[h]
\centering
\begin{tabular}{ccccc}
\toprule
& Spin & Charge & Mass & Force \\
\midrule
\Pphoton & 1 & 0 & 0 & EM \\
\PWpm & 1 & $\pm1$ & \SI{80.4}{\giga\electronvolt\per c^2} & Weak \\
\PZ & 1 & 0 & \SI{91.2}{\giga\electronvolt\per c^2} & Weak \\
\Pgluon & 1 & 0 & 0 & Strong\\
\PHiggs & 0 & 0 & \SI{125}{\giga\electronvolt\per c^2} & - \\
\bottomrule
\end{tabular}
\caption{The fundamental bosons of the Standard Model. The Higgs boson, \PHiggs, does not mediate a force but rather is required for the \PW and \PZ bosons to have mass under electroweak symmetry breaking.\label{tab:boson}}
\end{table}

Quarks may combine to form hadrons that are non-fundamental bosons, e.g.~$J^P(\Ppiplus)=0^-$.

\section{Forces}
At our energy scale there are four fundamental forces of nature: electromagnetism (EM), weak, strong, and gravity.

Gravity does not currently have a quantum formulation. Quantum gravity (QG) is in its theoretical infancy and, as shown below, experiments cannot reach the energy scales required for gravity to manifest itself at subatomic scales.

\subsection{Gravity}
Our best understanding of gravity -- that of general relativity (GR) -- describes the force as a result of the geometrical properties of spacetime. As such, GR predicts the existence of gravitational waves; ripples in spacetime emerging from extreme astrophysical situations causing large masses to oscillate. Gravitational waves have not yet been directly observed, but observations from Hulse and Taylor over a 17 year period observed a change in the period of a binary neutron star system. This change can only be accounted for by gravitational radiation, thus the experiment is an indirect detection of gravitational waves. This work won the Nobel Prize in 1993.

Gravity is known to be by far the weakest of the four fundamental forces. It is thought to be mediated by a spin-2 boson, the graviton. The strength of a force is proportional to the exchange momentum of the mediating boson, so very large energies will be required to observe gravity in a collider.

For a theory of everything (TOE), there is some energy scale at which gravity will unify with the Standard Model describing EW and QCD interactions. This is called the Planck mass and is derived here from the uncertainty principle.

\subsubsection{The Planck mass}
According to the uncertainty principle, a particle with mass $m$ can exist for time
\begin{equation}
\delta t \sim \frac{\hbar}{mc^2}
\end{equation}
with a characteristic range
\begin{equation}
\delta r = \delta t \times c = \frac{\hbar}{mc}.
\end{equation}
At this separation from the point mass, the gravitational potential energy between two particles with mass $m$ is
\begin{equation}
V = \frac{Gm^2}{\delta r} = \frac{Gm^3c}{\hbar}.
\end{equation}
When this potential energy is similar to a particle's mass energy, then gravitational effects become important and a theory of quantum gravity is needed. This occurs when $V \sim mc^2$, i.e.~
\begin{equation}
\frac{Gm^3c}{\hbar} \sim mc^2
\end{equation}
which is rearranged to define the Planck mass
\begin{equation}\boxed{
m_\text{Planck} \equiv \sqrt{\frac{\hbar c}{G}}
}
\end{equation}
whose value is of the order \SI[retain-unity-mantissa = false]{1e19}{\giga\electronvolt\per c^2}. Similarly, we may define the Planck length $\lambda_\text{Planck} \sim \SI[retain-unity-mantissa = false]{1e-35}{\meter}$, and the Planck time $t_\text{Planck} \sim \SI[retain-unity-mantissa = false]{1e-43}{\second}$.

When calculating integrals in the Standard Model (such as when evaluating loops in Feynman diagrams), the upper limit of integration should be the Planck scale, since this is where our understanding of nature breaks down.

\subsubsection{Dark matter}
Dark matter is believed to constitute about 25\% of the matter-energy content of the universe. This comes from observations of the rotational velocity curve of galaxies.

A simple model of a galaxy has a uniformly distributed spherical core with mass density $\rho$ at radii below $R$ and zero above. At gravitational equilibrium inside the galaxy, the centripetal force on some test star with mass $m$ is equal to the gravitational pull from the core inside the shell at distance $r$ from the centre,
\begin{align}
\frac{mv^2}{r} &= \frac{GmM(r)}{r^2} \\
\Rightarrow \quad v^2 &= \frac{GM(r)}{r} \\
&= \frac{GM_\text{galaxy}}{r}\left(\frac{r}{R}\right)^3\\
\Rightarrow \quad v &= \sqrt{\frac{GM_\text{galaxy}}{R^3}} \, r.
\end{align}
Outside the galaxy core we expect
\begin{equation}
v = \sqrt{\frac{GM_\text{galaxy}}{r}}.
\end{equation}
The observed discrepancy is known as the galaxy rotation problem, where the mass distribution of the galaxy is calculated from its luminous regions. The solution to the problem predicts a halo of dark matter around the edge of the galaxy.

Candidates for dark matter include:
\begin{itemize}
\item Massive cool hadronic objects (MaCHOs). Brown dwarves are small, dark stars that have been observed via gravitational lensing. A large abundance of MaCHOs is unlikely due to the known baryonic makeup of the early universe. There is not a sufficient number of MaCHOs to account for the dark matter problem.
\item Neutrinos. Since the mass of the neutrino is almost zero, neutrinos form hot dark matter. They carry thermal energy away from hot areas of the universe and do not cluster as halos around galaxies.
\item Supersymmetric particles as cold dark matter. Cold dark matter (CDM) must be made up of heavy particles for little thermal movement. Supersymmetric particles are the favourite CDM candidates in light of observations such as the anisotropy of the early universe. Supersymmetry (SUSY) solves some problems of the Standard Model and provides a natural CDM candidate, a particle with mass $\sim \SI{100}{\giga\electronvolt/c^2}$. These heavy particles are known as WIMPs (weakly interacting massive particles).
\item Light supersymmetric particles (LSPs) or hidden sector. Not yet explored area of particle physics could yield CDM candidates. There could be less massive particles that interact very weakly with ordinary matter; this is called the hidden sector. Additional theories of supersymmetry could predict light weakly interacting particles.
\end{itemize}

\subsection{Electromagnetism}
The electromagnetic (EM) force is described in the Standard Model by quantum electrodynamics (QED). All EM interactions involve the exchange of a virtual photon. The strength of the electromagnetic interaction is given by\begin{equation}
\frac{e^2}{q^2}
\end{equation}
where $q$ is the virtual photon's 4-momentum. $e$ is the electromagnetic coupling constant and is related to the fine structure constant by
\begin{equation}
e = \sqrt{4\pi\alpha}.
\end{equation}
The fine structure constant is itself defined as the ratio of the energy to overcome electrostatic repulsion between two electrons a distance $r$ apart and the energy of a photon of wavelength $\lambda = 2\pi r$.,
\begin{equation}
\alpha = \frac{e^2}{4\pi \epsilon_0 r} \left/ \frac{hc}{\lambda} \right. = \frac{e^2}{4\pi \epsilon_0 r} \left/ \frac{\hbar c}{r} \right. = \frac{e^2}{4\pi \epsilon_0 r \hbar c}.
\end{equation}
In natural units, where $\hbar = c = \epsilon_0 = 1$, this becomes
\begin{equation}
\boxed{\alpha = \frac{e^2}{4\pi}}.\label{eq:fineStructure}
\end{equation}


\subsection{The weak force}
The Fermi theory of weak interactions treats the nuclear beta decay as a single four-point vertex with strength $g_F$. The reaction is
\begin{equation}
\HepProcess{\Pneutron \to \Pproton + \Pelectron + \APnue}.
\end{equation}
A more sophisticated treatment recognises the role of the virtual W boson being exchanges as quarks change flavor,
\begin{equation}
\HepProcess{\Pdown \to \Pup + \PWminus \to \Pup + \Pelectron + \APnue}.
\end{equation}

In contrast to the EM interaction above, the strength of an interaction involving the exchange of the massive \PW boson is
\begin{equation}
\frac{g_W^2}{q^2 + m_W^2}
\end{equation}
where $g_W$ is similar to the electromagnetic $e$. This means, at energies much less than the \PW mass, $q^2 \ll m_W^2$, the weak force is much weaker than the electromagnetic force. However, for energies near to and above the \PW mass, $q^2 \geq m_W^2$, the forces are comparable and unify at energies around \SI{100}{\giga\electronvolt} or greater.

To first order, individual lepton numbers (i.e.~electron, muon, tau number) are conserved in weak interactions. This is equivalent to saying there is no mixing in the lepton sector, since leptons do not experience the strong force. Some example weak processes follow.
\begin{align}
\HepProcess{\Pnue + \Pneutron \to \Pproton + \Pelectron} \\
\HepProcess{\APnue + \Pproton \to \Pneutron + \Ppositron} \\
\HepProcess{\Pmuon \to \Pelectron + \Pnum + \APnue}
\end{align}

\subsection{The strong force}
Protons and neutrons are confined to the nucleus, so there must be a force which overcomes the electrostatic repulsion of the positive EM charges and also acts on neutrons. In scattering experiments,
\begin{equation}
\Ppiplus + \Pproton \xrightarrow{\Delta^{++}(1238)} \Ppiplus + \Pproton
\end{equation}
a resonance peak was seen at \SI{1238}{\mega\electronvolt\per c^2}, corresponding to the $\Delta^{++}$ resonance with valence quark contents $\Pup\Pup\Pup$. Since all three quarks in the baryon have the same spatial wavefunction, flavor, and at least two must have the same spin projection, this violates the exclusion principle without further quantum numbers. Introducing a color quantum number, with three color charges, solves this problem and provides a mechanism responsible for nuclear structure and quark confinement. The extra quantum number also means that for every hadronic decay there are three possible final states, giving a three-times color degeneracy to those decay modes.

The $SU(3)$ gauge model of the strong force results in an octet of gluons, which can self-interact via three- and four-point vertices. These extra vertices mean there are many more QCD diagrams for a given process, and calculating amplitudes in QCD is, in general, quite difficult. Furthermore, the QCD coupling constant is large -- owing to its high strength versus the electromagnetic interaction -- so, in general, one cannot stop with the first-order diagram. However, the running of the strong coupling constant means it decreases at high energies, so at LHC energies it is possible to use perturbation theory with QCD processes.

\section{Local gauge invariance}
The Lagrangian of a gauge theory is invariant under local transformations belonging to a particular symmetry group. This gauge invariance gives rise to the forces in the Standard Model -- EW and QCD. Why nature insists on gauge invariance is unknown.

As an example, consider a scalar theory of electromagnetism. Local gauge invariance under transformations belonging to the group $U(1)$ requires the Lagrangian is invariant to a rotation in phase space,
\begin{equation}
\psi(x, t) \rightarrow e^{i\alpha(x)} \psi(x, t).
\end{equation}
The transformation is said to be \emph{local} becuase the phase $\alpha(x)$ is a general function of space. Now requiring $\delta\mathcal{L}=0$ leads to the introduction of a massless gauge field, $A^\mu$; this is the photon field.

Similarly, local gauge invariance under transformations belonging to the groups $SU(2)$ and $SU(3)$ gives rise to the weak and strong forces, respectively. In this approach, however, the gauge bosons are all required to be massless, contrary to observations.

\section{Spontaneous symmetry breaking}

The EM and weak interactions have been successfully unified. That is, they have been shown to be two manifestations of the same [$SU(2) \times U(1)$] electroweak (EW) force, which they become at a higher energy (around \SI{100}{\giga\electronvolt}\footnote{In fact, the phase transition occurs at the Higgs vacuum expectation value (VEV), about \SI{246}{\giga\electronvolt}. The strength of the EM and weak forces merge at around \SI{100}{\giga\electronvolt}, so they appear as similar forces.}). As the universe cooled below this electroweak unification energy, the distinct EM and weak forces we see at normal energies froze out of the more general EW force. The spontaneous symmetry breaking leading to the EM and weak forces was described by Glashow, Salam, and Weinberg (GSW). The mechanism that allows the \PW and \PZ bosons to have mass after the EW symmetry breaking is the Higgs mechanism, of which the Higgs boson is a result. The strong force is described by quantum electrodynamics (QCD) and has not yet been unified with the EW interaction.

\section{Grand unified theories}

A grand unified theory (GUT) would bring the EW and QCD interactions together under a grand unified group, $G$ or $SU(5)$. Such theories would have to restore the broken symmetry we observe between colorless leptons and colored quarks, allowing quarks to turn into antileptons and vice versa. These interactions would violate conservation of baryon number, $B$, but $B-L$ (where $L$ is lepton number) would be conserved. The exchange boson is called the $X$ boson, which would have a mass of order \SI[retain-unity-mantissa = false]{1e6}{\giga\electronvolt\per c^2} and hence a range $10^{-4}$ times the weak force's.

This GUT force hence predicts the decay of the proton, which has not yet been observed although current experiments are attempting to measure the lifetime of the proton, if it is indeed unstable. An example decay is \HepProcess{\Pproton \to \Ppositron + \Ppizero}.
