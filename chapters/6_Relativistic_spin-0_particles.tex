\chapter{Relativistic spin-0 particles}
\section{The Klein-Gordon equation}
The Schr{\"o}dinger equation is the quantum mechanical equivalent of the classical $E = p^2/2m$. Now we want a relativistic version. Start from the relationship between energy, momentum, and mass,
\begin{equation}
E^2 - p^2 = m^2.
\end{equation}
Replacing the appropriate values with operators,
\begin{equation*}
E \rightarrow i \pdv{t}, \quad p \rightarrow -i \vec{\nabla}
\end{equation*}
and applying them to some general wavefunction,
\begin{equation}
\left( -\pdv[2]{t} + \nabla^2 \right)\psi = m^2 \psi.
\end{equation}
Rearranging and using equation \eqref{eq:nablaSquared} gives the Klein-Gordon equation
\begin{equation}\boxed{
\left( \partial^2 + m^2 \right)\psi = 0\label{eq:KleinGordon}
}.
\end{equation}
This is the fully relativistic equation of motion for spin-0 particles.

\subsection{4-current density}
We now wish to derive the probability 4-current from the Klein-Gordon equation. Taking the complex conjugate of \eqref{eq:KleinGordon} and multiplying by $\psi$ gives
\begin{equation}
\psi \left( \partial^2 + m^2 \right)\psi^* = 0\label{eq:KG1}.
\end{equation}
Similarly, multiplying \eqref{eq:KleinGordon} by $\psi^*$ gives
\begin{equation}
\psi^* \left( \partial^2 + m^2 \right)\psi = 0\label{eq:KG2}.
\end{equation}
Subtracting \eqref{eq:KG1} from \eqref{eq:KG2} and multiplying by $i$,
\begin{align}
i\psi^*\partial^2\psi - i\psi\partial^2\psi^* = 0 \\
\Rightarrow\quad \pdv{t} \left( i\psi^*\pdv{\psi}{t} - i\pdv{\psi^*}{t}\psi \right) - \vec{\nabla}\cdot\left( i\psi^* \vec{\nabla}\psi - i(\vec{\nabla}\psi)\psi^* \right) = 0
\end{align}
This result is a continuity equation,
\begin{equation}
\partial_\mu j^\mu = 0
\end{equation}
for the 4-current density
\begin{equation}
j^\mu = i\psi^* \partial^\mu \psi - i\psi \partial^\mu \psi^*
\end{equation}

\subsection{Application to a plane wave}
Consider a particle with wavefunction $\phi = \mathcal{N}e^{-iPX}$. Applying the above definition of 4-current density gives the values
\begin{align}
\rho = j^0 = 2\abs{\mathcal{N}}^2 \, E  \\
\vec{j} = 2\abs{\mathcal{N}}^2 \, \vec{p}.
\end{align}
That $\rho$ is proportional to $E$ is to be expected. Under a Lorentz boost the volume element transforms as
\begin{equation*}
\dd[3]{\vec{x}} \rightarrow \frac{\dd[3]{\vec{x}}}{\gamma}
\end{equation*}
so in order to preserve $\rho \dd[3]{\vec{x}}$, $\rho$ must transform as $\rho\rightarrow \gamma\rho$, in the same way energy does.

If the covariant nomalization to $2E$ particles per unit volume is used, the result is that $\rho = 1$.

\section{Negative energy particles and the Fenyman-Stueckelberg interpretation}
For the Klein-Gordon equation to be an accurate description of spin-0 particles, it should be able to describe antiparticles. Indeed, the Klein-Gordon equation does allow for particles with negative energy, i.e.~the negative solution of
\begin{equation}
E_\pm = \pm\sqrt{p^2 + m^2}.
\end{equation}
There are some problems associated with this. Firstly, the existence of negative energy states means that the energy of a particle can always be lowered. Secondly, the $E_-$ solutions are associated with a negative probability density $\rho$, which doesn't make sense and is not allowed.

Pauli and Weisskopf showed that it is possible to have a negative energy solution of the Klein-Gordon equation if the scalar electron charge $-e$ is included in $j^\mu$ and it is interpreted as a charge-current density,
\begin{equation}
j^\mu = -ie\left(\psi^* \partial^\mu \psi - i\psi \partial^\mu \psi^*\right).
\end{equation}
Now $\rho=j^0$ represents the charge density which is allowed to be negative.

Now that the theory has been fixed to allow the $E_-$ particles, we need an interpretation for them. The Feynman-Stueckelberg interpretation states that the negative energy states correspond to antiparticles. Consider the 4-charge-current of a scalar electron with a plane wavefunction,
\begin{equation}
j^\mu (\Pelectron) = -2e\abs{\mathcal{N}}^2\, \mqty(E \\ \vec{p}).
\end{equation}
In the same way, a positron has the current
\begin{align}
j^\mu (\Ppositron) &= 2e\abs{\mathcal{N}}^2\, \mqty(E \\ \vec{p})\\
&= -2e\abs{\mathcal{N}}^2\, \mqty(-E \\ -\vec{p}).
\end{align}
This is the same as for an electron with $-E$, $-\vec{p}$. So far as a system is concerned, the emission of a positron with energy $E$ is the same as the absorption of an electron with energy $-E$. In other words, positive-energy antiparticles travelling forwards in time are negative-energy particles travelling backwards in time. This gives the $E_-$ solutions an interpretation.
